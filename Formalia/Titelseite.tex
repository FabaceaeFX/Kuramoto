\thispagestyle{empty}
\begin{titlepage}
\newgeometry{top=0.5cm,bottom=0.1cm,right=1cm,left=1cm,bindingoffset=0.7cm}

\vspace*{0.7cm}

\begin{center}
    {\huge \textsc{Technische Universität Berlin}}
\end{center}

\vspace*{1.5cm}

\begin{center}
    {\LARGE \textsc{Bachelor Thesis}}\\
    \vspace{2mm}
    {\large \textsc{Summer semester 2018}}\\
    \vspace{2.0cm}
    \noindent\rule{15cm}{0.5pt}\\
    \vspace{5mm}
    {\LARGE \textbf{Synchronization of Chimera States in}}\\
    \vspace{2mm}
    {\LARGE \textbf{Multiplex Networks of Logistic Maps}}\\
    \vspace{1mm}
    \noindent\rule{15cm}{0.5pt}
\end{center}

\vspace{2cm}

\begin{center}
\begin{tabular}{ l r }
\textit{Author:} & \textit{Supervisor:}\\[1mm]
\textcolor{vaukgreen}{Marius \scshape{Winkler}} \includegraphics[width=0.5cm]{Formalia/Schuh1.png} & Prof. Dr. Dr. h.c. Eckehard {\scshape{Schöll}}, PhD\\
Email: win.m.winkler@gmail.com & schoell@physik.tu-berlin.de\\
Student ID: 364106 &  \\ \\
\textit{Assessors:} & \textit{Further advisors:}\\[1mm]
Prof. Dr. Dr. h.c. Eckehard {\scshape{Schöll}}, PhD & Jakub {\scshape{Sawicki}}, MSc. \\ 
Prof. Dr. Kathy \scshape{Lüdge} & Dr. Iryna \scshape{Omelchenko}\\
 & Dr. Anna \scshape{Zakharova}\\[1mm]

\end{tabular}
\end{center}


\vspace{1.0cm}


\begin{center}
\begin{tabular}{ l c r }
\includegraphics[width=4cm]{Formalia/tu-logo-2.png} & \hspace{7cm} & \includegraphics[width=2cm]{Formalia/sfb910.png}
\end{tabular}
\end{center}

\vspace{0.5cm}

\begin{center}
\includegraphics[width=3cm]{Formalia/fancy_Datum.png}
\end{center}

\end{titlepage}


\newpage\leavevmode\thispagestyle{empty}\newpage


\thispagestyle{empty}

%EIDESSTATTLICHE ERKLÄRUNG

\begin{figure}[b]
    {\Large \textbf{Eidesstattliche Erklärung}}\\
    \vspace{0.1cm}\\
    Hiermit erkläre ich, \textcolor{vaukgreen}{Marius \scshape{Winkler}}, dass ich die vorliegende Arbeit mit dem Titel
    \textit{\glqq Synchronization of chimera states in multiplex networks of logistic maps\grqq}~selbstständig
    und eigenhändig sowie ohne unerlaubte fremde Hilfe und ausschließlich
    unter Verwendung der aufgeführten Quellen und Hilfsmittel angefertigt habe.\\
    \vspace{0.1cm}\\
    \uline{Berlin, den \hfill}
\end{figure}


\newpage\leavevmode\thispagestyle{empty}\newpage
\newpage\leavevmode\thispagestyle{empty}\newpage


\thispagestyle{empty}

%ABSTRACT ENGLISCH

\begin{figure}
    \centering
    {\large \textsc{Technische Universität Berlin}}\\
    \vspace{0.5cm}
    {\huge \emph{Abstract}}\\
    \vspace{0.2cm}
    Fakultät II \\
    Institut für Theoretische Physik \\
    \vspace{0.3cm}
    \textit{Bachelor of Science}\\
    \vspace{0.3cm}
    \textbf{Synchronization of chimera states in multiplex networks of logistic maps}\\
    \vspace{0.2cm}
    by \textcolor{vaukgreen}{Marius \scshape{Winkler}}\\
\end{figure}
\noindent The last decade has seen an increasing interest in chimera states, complex spatio-temporal patterns combining coexisting coherent and incoherent domains. It was shown that they are not limited to phase oscillators but can be found in a large variety of different systems including time-discrete maps and time-continuous chaotic models, neural systems and autonomous Boolean networks. Due to the great importance of chimera states in complex systems it is an urgent topic to investigate the influence of chimera states in multiplex networks and their synchronizability. Relay synchronization via a relay-layer in between two outer layer can be useful for secure communication and modeling of brain dynamics.

In this present work beginning with theoretical studies of the logistic map dynamics, we investigate the behavior of chimera states within a one-dimensional ring topology. We discuss the coherence-incoherence bifurcation of coupled chaotic logistic maps with the appearance of chimera states by means of snapshots, space-time plots and the local order parameter $R_i$. We construct a 3-layer-multiplex network with the equivalent topologies as considered before in each layer and an additional inter-layer coupling. An analytic ansatz provides an indication for the critical intra-layer coupling strength $\sigma_c$ for the onset of chimeras in the more complex network. Different synchronization scenarios can be discovered and relay synchronization will appear as a not negligible but weak phenomenon. 

%Complex networks of multiple interacting layers can represent networks like Facebook friendships, airport network connected by airplanes or the neural brain network. 
\newpage\leavevmode\thispagestyle{empty}\newpage

\thispagestyle{empty}

%ABSTRACT DEUTSCH

\begin{figure}
    \centering
    {\large \textsc{Technische Universität Berlin}}\\
    \vspace{0.5cm}
    {\huge \emph{Kurzfassung}}\\
    \vspace{0.2cm}
    Fakultät II \\
    Institut für Theoretische Physik \\
    \vspace{0.3cm}
    \textit{Bachelor of Science}\\
    \vspace{0.3cm}
    \textbf{Synchronisation von Chimären-Zuständen in Multiplex-Netwerken der logistischen Gleichung}\\
    \vspace{0.2cm}
    von \textcolor{vaukgreen}{Marius \scshape{Winkler}}\\
\end{figure}
\noindent Seit ihrer Entdeckung sind Chimären Zustände im letzten Jahrzehnt immer weiter ins Zentrum des wissenschaftlichen Interesses gerückt. Chimären Zustände sind komplexe raum-zeitliche Strukturen, welche sowohl kohärente als auch inkohärente Bereiche aufweisen. Es konnte gezeigt werden, dass diese nicht nur bei Phasenoszillatoren auftreten, sondern sich in großer Vielfalt in verschiedenen Systemen wie zeitdiskreten Dynamiken und zeitkontinuierlichen chaotischen Modellen, sowie neuronalen Systemen und autonomen Boole'schen Netzwerken zeigen. Durch die große Bedeutung von Chimären-Zuständen in komplexen Netzwerken ist deren Einfluss auf Multiplex-Netzwerke und deren Synchronisierung eine entscheidende Fragestellung. Ein tieferes Verständnis über Relais Synchronisation zweier Ebenen, die indirekt über eine Überträgerebene miteinandern kommunizieren, könnte für die Gehirnforschung oder für sichere Kommunikation in unserem täglichen Leben von großem Interesse sein.

Diese Arbeit beginnt mit einer theoretischen Auseinandersetzung mit der gewählten logistischen Gleichung bevor das Verhalten von Chimären Zuständen in einer eindimensionalen Ring Topologie untersucht wird. Mit Hilfe von Momentaufnahmen, Raum-Zeit Diagrammen und dem lokalen Ordnungsparameter $R_i$ kann die koherent-inkoherent Bifurkation der gekoppelten, chaotischen Oszillatoren der logistischen Gleichung und das Auftreten von Chimären Zuständen diskutiert werden. Final wird aus der zuvor behandelten Toplogie ein 3-Ebenen-Mulitplex Netzwerk mit äquivalenten Ring Topologien in jeder Ebene und einer zusätzlichen inter-Ebenen Kopplung konstruiert. Ein analytischer Ansatz liefert einen Anhaltspunkt für eine kritische intra-Ebenen Kopplungsstärke $\sigma_c$ für das Auftreten von Chimären im nun größeren Netzwerk. Verschiedene Synchronisation-Szenarien werden offengelegt und Relais Synchronisation tritt auf, wenn auch als ein schwacher Mechanismus.

\newpage\leavevmode\thispagestyle{empty}\newpage

\setcounter{tocdepth}{2}

{\small\tableofcontents}
\addtocontents{toc}{~\vspace{-2\baselineskip}}
\thispagestyle{empty}

\newpage\leavevmode\thispagestyle{empty}\newpage

%\pagebreak
\pagenumbering{arabic}\setcounter{page}{1}